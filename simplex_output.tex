% Archivo generado automáticamente por simplex.c
\documentclass[12pt]{article}
\usepackage[utf8]{inputenc}
\usepackage{graphicx}
\usepackage{array,booktabs}
\usepackage[table]{xcolor}
\usepackage{longtable}
\usepackage{geometry}
\usepackage{pdflscape}
\usepackage{tikz}
\usepackage{pgfplots}
\pgfplotsset{compat=1.18}
\geometry{margin=0.8in}
\title{Proyecto 4: SIMPLEX}
\author{Investigación de Operaciones}
\date{}
\begin{document}
\begin{center}
{\large Instituto Tecnológico de Costa Rica\\[1cm]
\includegraphics[width=0.4\textwidth]{TEC.png}\\[2cm]
{\LARGE \textbf{Proyecto 4: SIMPLEX}}\\[2cm]
{\large Investigación de Operaciones\\[2cm]
{\large Profesor: }\\[1cm]
{\large Francisco Jose Torres Rojas}\\[2cm]
{\large Integrantes: }\\[1cm]
{\large Jose Pablo Fernandez Jimenez - 2023117752}\\[1cm]
{\large Diego Durán Rodríguez - 2022437509}\\[2cm]
{\large Segundo semestre 2025\\[1cm]
\end{center}
\newpage
\section*{Algoritmo SIMPLEX}
El \textbf{algoritmo SIMPLEX} es uno de los métodos más importantes y utilizados en el campo de la optimización lineal. Fue desarrollado por George Bernard Dantzig en 1947, en el contexto de investigaciones relacionadas con la planificación logística y de recursos durante la posguerra. Dantzig, un matemático y científico estadounidense, ideó este método como una herramienta para resolver problemas de programación lineal, un área que busca optimizar (maximizar o minimizar) una función objetivo sujeta a un conjunto de restricciones lineales. 

El surgimiento del algoritmo Simplex marcó un antes y un después en la optimización matemática. Antes de su creación, no existía un procedimiento general y sistemático que permitiera resolver eficientemente problemas de gran escala con múltiples variables y restricciones. Dantzig propuso un enfoque geométrico basado en la observación de que la solución óptima de un problema lineal se encuentra en uno de los vértices o puntos extremos del poliedro factible, es decir, del conjunto de soluciones que cumplen todas las restricciones del problema. 

El método Simplex avanza de un vértice a otro a través de las aristas del poliedro, mejorando progresivamente el valor de la función objetivo hasta encontrar el óptimo. Cada movimiento corresponde a un cambio de una variable básica en la solución, lo que permite al algoritmo recorrer el espacio factible de manera ordenada y eficiente. 

\textbf{Entre las principales propiedades del algoritmo Simplex destacan las siguientes:}
\begin{itemize}
  \item \textit{Eficiencia práctica:} aunque en teoría su complejidad puede ser exponencial en el peor de los casos, en la práctica el algoritmo es extremadamente eficiente y puede resolver problemas con miles de variables y restricciones en tiempos incluso lineales.
  \item \textit{Interpretación geométrica clara:} El procedimiento del Simplex se basa en conceptos geométricos simples, lo que facilita su comprensión y visualización en espacios de baja dimensión.
  \item \textit{Importancia histórica y teórica:} el Simplex no solo revolucionó la programación lineal, sino que también sentó las bases para la aparición de otros métodos de optimización, como los algoritmos de punto interior y técnicas modernas de optimización convexa.
\end{itemize}

\bigskip
\section*{Problema original}
Nombre del problema: \textbf{Problema 1 P1}\\[0.3cm]
El problema original se puede formular como un problema de programación lineal, donde se busca optimizar una función objetivo sujeta a ciertas restricciones:

\\[0.5cm]
\vspace{0.4cm}
\textbf{Maximizar} \textbf{Z = } $X_4$\\[0.5cm]
\textbf{Sujeto a:}\\
5 $X_1$ + $X_2$ + 4 $X_3$ - $X_4$ \geq 0 \\
6 $X_2$ + 4 $X_3$ - $X_4$ \geq 0 \\
6 $X_1$ + 4 $X_2$ + 8 $X_3$ - $X_4$ \geq 0 \\
$X_1$ + $X_2$ + $X_3$ = 1 \\
\\[0.3cm]
\textbf{Con } $X_1$, $X_2$, $X_3$, $X_4$ \geq 0\\[0.5cm]
\bigskip
\section*{Tabla inicial}
\begin{center}
\rowcolors{2}{white}{gray!10}
\begin{tabular}{rrrrrrrrrrrrr}
\toprule
\textbf{$Z$} & \textbf{$X_1$} & \textbf{$X_2$} & \textbf{$X_3$} & \textbf{$X_4$} & \textbf{$e_{1}$} & \textbf{$e_{2}$} & \textbf{$e_{3}$} & \textbf{$a_{1}$} & \textbf{$a_{2}$} & \textbf{$a_{3}$} & \textbf{$a_{4}$} & \textbf{$RHS$} \\\midrule
1 & -0 & -0 & -0 & -1 & 0 & 0 & 0 & M & M & M & M & 0\\
0 & 5 & 1 & 4 & -1 & -1 & 0 & 0 & 1 & 0 & 0 & 0 & 0\\
0 & 0 & 6 & 4 & -1 & 0 & -1 & 0 & 0 & 1 & 0 & 0 & 0\\
0 & 6 & 4 & 8 & -1 & 0 & 0 & -1 & 0 & 0 & 1 & 0 & 0\\
0 & 1 & 1 & 1 & 0 & 0 & 0 & 0 & 0 & 0 & 0 & 1 & 1\\
\bottomrule
\end{tabular}
\end{center}
\section*{Tabla final}
\begin{center}
\rowcolors{2}{white}{gray!10}
\begin{tabular}{rrrrrrrrr}
\toprule
\textbf{$Z$} & \textbf{$X_1$} & \textbf{$X_2$} & \textbf{$X_3$} & \textbf{$X_4$} & \textbf{$e_{1}$} & \textbf{$e_{2}$} & \textbf{$e_{3}$} & \textbf{$RHS$} \\\midrule
1 & 0 & 2 & 0 & 0 & 0.8 & 0.2 & 0 & 4\\
0 & 0 & 2 & 1 & 0 & 0.2 & -0.2 & 0 & 1\\
0 & 0 & 2 & 0 & 1 & 0.8 & 0.2 & 0 & 4\\
0 & 1 & -1 & 0 & 0 & -0.2 & 0.2 & 0 & -2.22045e-16\\
0 & 0 & 4 & 0 & 0 & -0.4 & -0.6 & 1 & 4\\
\bottomrule
\end{tabular}
\end{center}
\section*{Solución}
$Z = 4$\\[0.3cm]
\section*{Valores de todas las variables}
\begin{tabular}{lr}\toprule Variable & Valor \\ \midrule
$X_1$ & -2.22045e-16 \\
$X_2$ & 0 \\
$X_3$ & 1 \\
$X_4$ & 4 \\
e_{1} & 0 \\
e_{2} & 0 \\
e_{3} & 4 \\
\bottomrule\end{tabular}\\[0.3cm]
\subsection*{Problema degenerado}
Se detectó degeneración durante la ejecución del Simplex. \newline
Definición: una solución básica factible es degenerada cuando alguna variable básica tiene valor cero. Esto puede ocurrir cuando en el test de razón mínima la razón mínima es 0 o cuando hay un empate entre razones.\\[0.2cm]
El programa marca como degeneración cuando dos razones difieren en menos de un epsilon (1e-09) o cuando la razón mínima es 0. Para romper el empate se adopta una heurística simple: se selecciona la primera fila con la razón mínima encontrada. Esta elección se documenta en las tablas intermedias (fila pivote seleccionada).\\[0.2cm]
\renewcommand{\refname}{Referencias}
\begin{thebibliography}{9}
\bibitem{wikipedia2025} Wikipedia contributors. (2025, 6 octubre). Simplex algorithm. \textit{Wikipedia}. Disponible en: https://en.wikipedia.org/wiki/Simplex_algorithm

\bibitem{benlowery2022} Ben-Lowery. (2022, 4 abril). Linear Programming and the birth of the Simplex Algorithm. \textit{Ben Lowery @ STOR-i}. Disponible en: https://www.lancaster.ac.uk/stor-i-student-sites/ben-lowery/2022/03/linear-programming-and-the-birth-of-the-simplex-algorithm/

\bibitem{libretexts2022} Libretexts. (2022, 18 julio). 4.2: Maximization by the Simplex method. \textit{Mathematics LibreTexts}. Disponible en: https://math.libretexts.org/Bookshelves/Applied_Mathematics/Applied_Finite_Mathematics_%28Sekhon_and_Bloom%29/04%3A_Linear_Programming_The_Simplex_Method/4.02%3A_Maximization_By_The_Simplex_Method

\end{thebibliography}
\end{document}
